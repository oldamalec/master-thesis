\chapter{Zadání úkolů pro testery alfa verze aplikace} \label{ap:testing_tasks}

\begin{enumerate}
	\item Práce rozdělená na vedoucího a skladníka – naskladnění
	\begin{itemize}
		\item Na PC nebo mobilu se přihlaste jako test\_vedouci / test\_vedouci.
		\item Přišel Vám nový produkt, který ještě není v jiných systémech. Založte mu novou skladovou kartu:
		\begin{itemize}
			\item vyplňte údaje, které běžně u skladové karty potřebujete,
			\item přidejte i jeho čárový kód,
			\item tento produkt je někdy i zabalen v kartonu, ve kterém jsou 4 kusy.
		\end{itemize}
		\item Vytvořte nový úkol „naskladnění“, čímž zaúkolujete skladníka, aby přijal toto nové zboží.
		\item Na mobilu se přihlaste jako test\_skladnik / test\_skladnik.
		\item Najděte před chvílí vytvořený úkol naskladnění a začněte na něm pracovat.
		\item Naskladněte několik různých kusů výrobků na více umístění (podle konkrétního zadání) – použijte k tomu čtečku kódů.
		\item Po dokončení úkolu se odhlašte.
		\item Přihlaste se jako vedoucí, a zkontrolujte tento úkol, tedy že naskladněné zboží odpovídá reálné situaci ve skladě, pokud ano, úkol schvalte, jinak ho vraťte a za skladníka chybu opravte.
	\end{itemize}
	\item Práce vedoucího a skladníka – vyskladnění
	\begin{itemize}
		\item Jako vedoucí založte nový úkol vyskladnění
		\item Způsob předání zvolte \uv{Expedice}
		\item Zadejte zboží k vyskladnění podle zadání (bude upřesněno)
		\item Jako skladník tento úkol přijměte a posbírejte požadované zboží k sobě, zatím ho ale nepřemisťujte na umístění určené k expedici.
		\item Vraťte se na seznam úkolů a dejte si oběd.
		\item Po pauze na oběd se vracíte k práci, a úkol vyskladnění, jehož zboží máte stále u sebe, chcete dokončit – proveďte to.
		\item Po dokončení se přihlaste jako vedoucí a zkontrolujte, kolik bylo na úkolu stráveno času.
	\end{itemize}
	\item Práce vedoucího a skladníka v jedné osobě – naskladnění
	\begin{itemize}
		\item Přihlaste se jako test\_master / test\_master
		\item Znovu vytvořte úkol naskladnění, dle zadání (bude upřesněno)
		\item Rovnou po vytvoření začněte na úkolu pracovat, a dokončete ho bez nutnosti schvalování vedoucím – protože vy jste vedoucí.
	\end{itemize}
	\item Přesun mezi umístěními
	\begin{itemize}
		\item Zvolte si, zda budete úkol zadávat skladníkovi, nebo použijete uživatele s oběma rolemi.
		\item Vytvořte úkol \uv{Přesun položek} – dle konkrétního zadání (bude upřesněno)
		\item Úkol realizujte a dokončete
	\end{itemize}
\end{enumerate}
