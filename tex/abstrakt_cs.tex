Práce se zaměřuje na kompletní proces analýzy, návrhu, vývoje a testování frontendu webové aplikace, s důrazem na praktické použití. Nový systém staví na požadavcích již existující aplikace a klade si za cíl implementovat veškeré její funkcionality ale také další rozšíření, která již není efektivní zapracovávat do staré verze. Návrh uživatelského rozhraní aplikace staví na projektu z předmětu MI-NUR vyučovaného na FIT ČVUT v Praze. Výsledná reálná aplikace využívá Javascriptový framework Vue.js a grafickou knihovnu Vuetify. Frontendová část aplikace, která je předmětem této práce, implementuje REST API vytvořené v souběžné diplomové práci. V rámci práce byl vytvořen základ nového skladového systému, na kterém je možné dále stavět a rozvíjet jeho funkcionality. Z texty lze vycházet při návrhu obdobných aplikací, zejména co se technologické volby týká.
