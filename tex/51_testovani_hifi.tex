\section{Testování prototypu}

Hi-Fi prototyp, který je dostupný na přiloženém médiu, byl testován v rámci předmětu MI-NUR společně mnou, Pavlem Kovářem, Martinem Kubišem a Jakubem Šterclem. Abych nekradl cizí práci, uvedu v tomto textu pouze části, které jsou realizoval já osobně, i výstup ostatních kolegů jsem ale zhodnotil v samotné implementaci následné funkční aplikace.

\subsection{Heuristická analýza}

Zvolili jsme několik akcí skladníka, které jsme podrobili heuristické analýze. Já konkrétně jsem řešil obrazovku naskladnění (screenshot přiložen v předchozí kapitole jako obrázek \ref{picture:hifi}), která dopadla následovně:

\begin{enumerate}
	\item \emph{Viditelnost stavu systému}:\\Vše v pořádku
	\item \emph{Shoda mezi systémem a realitou}:\\OK
	\item \emph{Minimální zodpovědnost (a stres)}:\\Při otevřené tohoto tasku se mu ten hned přiřadí. Pro navrácení je nutné jej stornovat a vyplnit, že se jednalo o chybu. V systému o tom však bude záznam. Dále není jasné, zda tlačítka ve spodní části úkol již odeslou, nebo se teprve zobrazí nějaký formulář a potvrzení.\footnote{V reálné aplikace je toto vyřešeno: úkol při otevření - pokud ještě není přiřazen - zobrazí pouze informace a tlačítko pro přiřazení úkolu \uv{sám sobě}. Teprve poté je možné v něm provádět práci. Tlačítka ve spodní části byla předělána na běžná tlačítka a byla přidána další obrazovka, kde se úkol dokončuje. Zda tlačítko něco odesílá, řeší rozlišení jejich barev, což jsem popisoval v sekci \ref{implementation:colors:idempotent}.}
	\item \emph{Shoda s použitou platformou a obecnými standardy}:\\V pořádku (material design)
	\item \emph{Prevence chyb}:\\Úkol nelze dokončit, pokud nějaké zboží nemá vyplněno umístění.\footnote{V reálné aplikaci toto řešeno výrazným označením zboží, které ještě není umístěno.} Jiné chyby na této stránce (z aplikačního hlediska) prakticky nelze udělat => OK
	\item \emph{Kouknu a vidím}:\\Při větším množství položek může být seznam dlouhý a nepřehledný
	\item \emph{Flexibilita a efektivita}:\\OK - zkušený může stále pípat, nový může zkoumat dostupné vstupy.
	\item \emph{Minimalita (Klapky na očích)}:\\OK - obsah karet bude ve finální aplikaci konfigurovatelný\footnote{V době psaní textu se jedná o jeden z návrhů na budoucí vylepšení aplikace}.
	\item \emph{Smysluplné chybové hlášky}:\\Jediná chybová hláška o vyplnění umístění je jasná.
	\item \emph{Help a dokumentace}:\\Dokumentace není\footnote{Reálná aplikace základní nápovědu / dokumentaci nabízí}
\end{enumerate}

// TODO more
