Na těchto řádcích bych chtěl poděkovat všem, bez kterých by tato práce nemohla vzniknout, nebo by nebyla v odpovídající kvalitě. V první řadě se jedná zejména o vedoucího práce - Ing. Jiřího Hunku, který mi poskytl dostatečnou volnost při její tvorbě, ale zároveň zařídil nejdůležitější kroky k jejímu zdárnému vytvoření - zejména velkou zpětnou vazbu ve fázi analýzy a návrhu, a ke konci realizace například zařízení testerů v reálných skladech. Stejně důležité díky patří i mému kolegovi - Ing. Pavlu Kovářovi, který je autorem backendové části řešené aplikace, a bez jehož API by byla má práce pouze jakousi nefunkční šablonou - jak řekl klasik\footnote{Marek Erben} - \uv{díky frontendu je to hezké, ale díky backendu to funguje}. Pavlova dokumentace API tak pro mě byla dokonalým zdrojem informací, jaké funkce musím ještě pokrýt a jak se mají chovat a právě díky tomu jsem se při implementaci mohl soustředit opravdu pouze na frontendové řešení.\\
S počátky této práce mi velmi pomohl také tým z MI-NUR\footnote{Návrh uživatelského rozhraní - předmět na FIT.}, který se kromě mě skládal právě z Ing. Pavla Kováře, Ing. Martina Kubiše a také Bc. Jakuba Štercla - všem tímto moc děkuji, že se podíleli na návrhu frontendu mé diplomové práce. Prototyp, který v rámci MI-NUR vznikl, testovalo celkem osm osob, a přesto že já osobě jsem se setkal pouze s dvěma z osmi, všichni si zaslouží poděkování - konkrétně se jedná o Marka Erbena, Davida Zahrádku, Pavla Štercla, Milana Kubiše, Barbaru Novotnou, Kristýnu Miltnerovou, Václava Kováře a Lukáše Svobodu.\\
V závěru práce jsem opět oslovil několik testerů, kteří již testovali funkční aplikaci na reálném zařízení, v jejich přirozeném prostředí. I ti si zaslouží velké díky, avšak při testování jsme se domluvili, že nebudu zveřejňovat jejich jména. Pokud se jim však někdy tento text dostane do ruky, oni budou jistě vědět - díky!\\
Nelze opomenou ani jazykovou a stylistickou korekturu tohoto textu, za kterou vděčím Markétě Malcové a Kristýně Miltnerové, také děkuji!
