Evidence skladovaného zboží je snad jedním z nejčastějších požadavků na informační systém, je potřeba jak v malých skladech spravovaných jedním člověkem, až po obří sklady rostoucí na okrajích velkých měst. Od skladových systémů většina jeho uživatelů očekává především přehled počtů skladovaného zboží, evidenci umístění, ale také snadné napojení na externí aplikace. Aplikací řešících tuto problematiku existuje nepřeberné množství, avšak z důvodu potřeby nahradit existující skladovou aplikaci, do které je již nevhodné zahrnovat nově vzniklé požadavky, vznikl záměr vytvořit vše od podlahy.\\
Již v prvotním návrhu bylo rozhodnuto, že aplikace bude striktně rozdělena na serverovou a klientskou část, čímž vznikla dvě témata diplomových prací, z nichž klientskou část řeším právě v rámci tohoto textu. Má motivace pro vytvoření frontendové části nového skladového systému spočívá v zájmu o webové technologie, především pak moderní Javascriptové frameworky.\\
V textu projdu postupně všechny body standardního vývoje aplikace, počínaje analýzou požadavků a konkurenčních řešení, návrhem prototypu a jeho testováním, následně implementací reálné aplikace, na kterou se váže hojně diskutovaná volba technologií a popis některých klíčových částí kódu. Poslední kapitola obsahuje zápisy a postřehy z uživatelského testování, ze kterého vzešly relevantní požadavky na úpravu, z nich ty nejpalčivější jsem ještě v rámci této práce zapracoval.\\
Samotný text je hojně doplněn vhodnými přílohami, které obsahují rozpracované analýzy či úkoly pro testery a zápisy z testování.
