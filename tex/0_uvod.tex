Evidence skladovaného zboží je snad jedním z nejčastějších požadavků na informační systém - je potřeba jak v malých skladech spravovaných jedním člověkem, až po obří sklady, které rostou na okrajích velkých měst. Od skladových systémů většina jeho uživatelů očekává především přehled počtů skladovaného zboží, případně evidenci umístění, ale také snadné napojení na externí aplikace. Aplikací řešících tuto problematiku existuje nepřeberné množství, avšak z důvodu potřeby nahradit existující skladovou aplikaci, provozovanou společností ve které působím, a do které již není vhodné zahrnovat nové funkcionality, byl vytvořen záměr napsat skladový systém na čisto od podlahy.\\
Již v prvotním návrhu bylo rozhodnuto, že aplikace bude striktně rozdělena na serverovou a klientskou část, čímž vznikla dvě témata diplomových prací, z nichž klientskou část řeším právě v rámci tohoto textu. Má motivace pro vytvoření frontendové části nového skladového systému spočívá v zájmu o webové technologie a chuť naučit se pracovat s moderním Javascriptovým frameworkem.\\
V textu projdu postupně všechny body standardního vývoje aplikace, počínaje analýzou požadavků a konkurenčních řešení, návrhem prototypu a jeho testováním, přes implementaci reálné aplikace, která začíná dlouhou diskusí volby technologie a pokračuje popisem některých klíčových částí kódu. Poslední kapitola obsahuje zápisy a postřehy z uživatelského testování, ze kterého vzešly relevantní požadavky na úpravy, z nichž ty nejpalčivější jsem ještě v rámci této práce zapracoval.\\
Samotný text je hojně doplněn přílohami, které obsahují rozpracované analýzy, úkoly pro testery či zápisy z testování.
