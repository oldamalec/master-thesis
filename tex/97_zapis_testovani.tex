\chapter{Zápis z uživatelského testování aplikace během vývoje} \label{ap:testing_notes}

\section{Seznam požadavků}

\begin{itemize}
	\item Na klik na šedý button formuláře ukázat, co vlastně chybí vyplnit,
	\item výchozí akce tabulky by měla být aktivní na celém řádku,
	\item karty na domovské obrazovce řadit rovnou pod sebe - jako v Google Keep,
	\item tabulky jsou moc fádní, moc splývají,
	\item přidat k úpravě skladu, co znamená volba "malý sklad",
	\item přidat ke skladu atribut "popis",
	\item v malém skladu povolit více umístění,
	\item přidat skladníkovi možnost "Rychlý přesun mezi umístěními" ,
	\item na tlačítka která pouze něco rozbalují a nepřidávají, dát jinou ikonku než "+",
	\item doplnit popisky, že umístění je fyzické a části skladu jsou virtuální,
	\item přejmenovat "první skladová položka" a "první vlastník" na reálnější údaje,
	\item přidat možnost "vytvořit nové" i přímo do tabulek, nejen v rohu obrazovky,
	\item přidat do načítání dat o subjektu nejen základní data z ARES, ale i bankovní účet, plátce DPH, nespolehlivý plátce,
	\item ve formulářích, kde se vyplňuje i IČO, ho mít na první místě, aby to uživatel vyplňoval jako první a zbytek jen načetl,
	\item k vlastníkovi skladu přidat pole "oddělení",
	\item backend by měl vypočítávat IBAN z čísla účtu,
	\item zákazník má mít možnost mít více doručovacích adres,
	\item veškeré inputy typu textarea dát jako rámeček, ne jen linku,
	\item naskladnění - skladník:
	\begin{itemize}
		\item přidat ke kusům "ks",
		\item umožnit zobrazení většího náhledu fotku,
		\item výběr počtu štítků k tisku,
		\item řešit, že je zboží s výhradou (poškozený obal atp.),
		\item skladník možnost přidávat fotky,
		\item k položkám ukládat jednotky, ty pak vypisovat v naskladnění (ks, litrů, kg, ...),
	\end{itemize}
	\item naskladnění - vedoucí (přehled hotového):
	\begin{itemize}
		\item umožnit rychlý náhled výrobku,
		\item zobrazovat fotku dodáku,
		\item hledání v přehledu naskladněných položek,
	\end{itemize}
	\item v záložkách na domovské obrazovce zobrazovat počty úkolů v jednotlivých seznamech,
	\item zvýraznit priority (možná udělat barevně i ikonky),
	\item přidat vedoucímu možnost vytvořit samotné naskladnění (bez dodávky),
	\item inventura:
	\begin{itemize}
		\item našeptávat umístění k napípání,
		\item zabalit detaily umístění, které nejsou aktivní,
		\item přidat autoscroll na napípnuté umístění,
		\item přidat fotku výrobku,
		\item položky na umístění zalamovat pod sebe,
		\item vedle názvu zobrazovat i EAN a kolikakusový EAN to je,
		\item přidat stránku přehledu inventury, která bude mít stejný obsah, jako se tiskne do PDF reportu.
	\end{itemize}
\end{itemize}

\section{Seznam chyb}

\begin{itemize}
	\item nelze změnit název atributu (PUT mění jen value),
	\item v příjmu dodávky se nevypisuje, kdo ji přijal, je tam "nikdo",
	\item nelze dokončit naskladnění,
	\item při odmítnutí řešení úkolu se špatně odesílají poznámky.
\end{itemize}

\section{Seznam otázek ke zjištění}

\begin{itemize}
	\item Mají tabákové výrobky jiný EAN pro jiné šarže? (možná mění jen kolky),
	\item zjistit, jak se dá na dodáky tisknout datová informace o položkách (QR kód s linkem na api),
	\item jak se řeší vyskladňování šaržovaných výrobků.
\end{itemize}

\section{Návrhy na pokročilé funkce}

\begin{itemize}
	\item Kategorie produktů,
	\item rozpis vytíženosti skladu: k tomu je potřeba řešit počet pozic na umístění, blokace pozic,
	\item routování trasy skladníka: nejdřív naber velké věci, pak teprve ty malé,
	\item řešit umístění u země (upozorňovat, že jsou u země výrobky, které se netočí),
	\item řešit systémově výrobky, které zabírají více palet (např. 3),
	\item hromadný aktualizátor dat z Aresu s výběrem diffu,
	\item příprava na dodávky zboží: vedoucí může zadávat skladníkům avizace dodávek = příprava následujícího úkolu naskadnění,
	\item skupiny "typů" skladníků + možnost pro vedoucího přiřazovat úkoly přímo skladníkům / skupinám skladníků,
	\item pokročilé nastavení domovské obrazovky: možnost nastavení množství informací na kartách úkolů,
	\item víceúrovňový konfigurovatelný Dashboard,
	\item nový úkol "kompletní inventura skladu": vychází z dokončené jiné inventury, nabízí k inveturizaci pouze to, co je např. jen na špatném umístění, zobrazuje, kde se chybějící zboží pohybovalo,
	\item nastavování práv skladníků:
	\begin{itemize}
		\item možnost tvořit nové položky?,
		\item možnost tvořit nová umístění?,
		\item pokud nemá právo a potřebuje to udělat, tak možnost "zavolej vedoucího".
	\end{itemize}
\end{itemize}
