\newglossaryentry{gBackend}{
	name=Backend,
	description={
		Část aplikace, která se stará o ukládání dat, jejich zpracování a bussiness logiku. Většinou není přímo přístupná koncovému uživateli, ten k ní přistupuje přes frontend.
	}
}
\newglossaryentry{gDepInj}{
	name=Dependency Injection,
	description={
		Technika, která umožňuje "vložení" instance objektu, který poskytuje nějakou službu, do jiného objektu, který pak může danou službu efektivně používat.
	}
}
\newglossaryentry{gFramework}{
	name=Framework,
	description={
		Softwarová struktura, která slouží jako podpora pro pohodlnější programování. Může obsahovat podpůrné funkce, knihovny či nástroje pro efektivnější, bezpečnější a pohodlnější vývoj softwaru.
	}
}
\newglossaryentry{gFrontend}{
	name=Frontent,
	description={
		Část aplikace, s kterou přímo interaguje koncový uživatel či administrátor, typicky pomocí GUI. Většinou komunikuje s druhou, serverovou částí - backendem.
	}
}
\newglossaryentry{gGitHub}{
	name=GitHub,
	description={
		GitHub je webová služba podporující vývoj softwaru za pomoci verzovacího nástroje Git.
	}
}
\newglossaryentry{gMiddleware}{
	name=Middleware,
	description={
		Software realizující integraci mezi dvěma jinými systémy, typicky pomocí API.
	}
}
\newglossaryentry{gTypeScript}{
	name=TypeScript,
	description={
		Nadmnožina JavaScriptu, která jej rozšiřuje především o statické typování proměnných a další atributy z OOP.
	}
}
\newglossaryentry{gWebView}{
	name=WebView,
	description={
		Komponenta nativní Android aplikace, která zobrazuje stanovenou URL jako svůj obsah. Používá se zejména v místech, kde je žádoucí zobrazovat obsah z webu, ale je potřeba přístup k funkcím zařízení, ke kterým není možné přístupovat z běžného webového prohlížeče.
	}
}
