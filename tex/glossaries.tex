\newglossaryentry{gBackend}{
	name=Backend,
	description={
		Část aplikace, která se stará o ukládání dat, jejich zpracování a bussiness logiku. Většinou není přímo přístupná koncovému uživateli, ten k ní přistupuje přes frontend.
	}
}
\newglossaryentry{gDepInj}{
	name=Dependency Injection,
	description={
		Technika, která umožňuje "vložení" instance objektu, který poskytuje nějakou službu, do jiného objektu, který pak může danou službu efektivně používat.
	}
}
\newglossaryentry{gECMAScript}{
	name=EcmaScript,
	description={
		Definice programovacího jazyka, kterou implementuje například Javascript. 
	}
}
\newglossaryentry{gFavicon}{
	name=Favicon,
	description={
		Ikonka webové stránky, která se ve většině prohlížečů zobrazuje v panelu v horní liště a na dalších místech, které odkazují na daný web.
	}
}
\newglossaryentry{gFramework}{
	name=Framework,
	description={
		Softwarová struktura, která slouží jako podpora pro pohodlnější programování. Může obsahovat podpůrné funkce, knihovny či nástroje pro efektivnější, bezpečnější a pohodlnější vývoj softwaru.
	}
}
\newglossaryentry{gFrontend}{
	name=Frontent,
	description={
		Část aplikace, s kterou přímo interaguje koncový uživatel či administrátor, typicky pomocí GUI. Většinou komunikuje s druhou, serverovou částí - backendem.
	}
}
\newglossaryentry{gGit}{
	name=Git,
	description={
		Git je distribuovaný verzovací nástroj určený zejména pro sdílení a verzování zdrojových kódů aplikací, ale i jiných assetů.
	}
}
\newglossaryentry{gGitHub}{
	name=GitHub,
	description={
		GitHub je webová služba podporující vývoj softwaru za pomoci verzovacího nástroje Git.
	}
}
\newglossaryentry{gHotSwap}{
	name=Hot Swapping,
	description={
		Jedná se o způsob zobrazení změn v aplikace při změně jejího kódu. Při hot-swappingu není potřeba spouštět žádné procesy sestavení aplikace a často ani načíst znovu obrazovku - vše je provedeno automaticky za běhu a změny jsou viditelné ihned.
	}
}
\newglossaryentry{gMiddleware}{
	name=Middleware,
	description={
		Software realizující integraci mezi dvěma jinými systémy, typicky pomocí API.
	}
}
\newglossaryentry{gSentry}{
	name=Sentry,
	description={
		Sentry je webová služba pro sledování chyb, které v aplikaci nastanou. Při použití v produkčním prostředí může vývojář díky Sentry o chybě vědět ještě dříve, než ji uživatel nahlásí, a to včetně všech detailů, jako například jaké kroky chybě předcházely, prostředí, ve kterém k chybě došlo a mnoho dalších.
	}
}
\newglossaryentry{gSnackbar}{
	name=Snackbar,
	description={
		Velmi podobný toast message, narozdíl od čistě informativního toast message umožňuje Snackbar uživateli spustit i programátorem definovanou akci. Zobrazuje se typicky ve spodní části aplikace a informuje o proběhlé akci.
	}
}
\newglossaryentry{gSubversion}{
	name=Subversion,
	description={
		Systém pro správu zdrojových kódů a dalších assetů. Dnes se používá zejména pro ne-textové soubory, ty textové jsou většinou verzovány v gitu.
	}
}
\newglossaryentry{gToastMessage}{
	name=Toast message,
	description={
		Krátká informativní zpráva, který se objevuje ve spodní části aplikace a obsahuje typicky informaci o potvrzení provedení akce.
	}
}
\newglossaryentry{gTrello}{
	name=Trello,
	description={
		Webová služba pro evidenci úkolů, s flexibilním nastavením sloupců, označení atp. ve stylu kanban.
	}
}
\newglossaryentry{gTypeScript}{
	name=TypeScript,
	description={
		Nadmnožina JavaScriptu, která jej rozšiřuje především o statické typování proměnných a další atributy z OOP.
	}
}
\newglossaryentry{gUseCase}{
	name=Use case,
	description={
		TODO
	}
}
\newglossaryentry{gVuetify}{
	name=Vuetify,
	description={
		Knihovna pro Vue.js, která implementuje Material Design a poskytuje základní stavební komponenty, ale i pokročilé bloky jako například datové tabulky.
	}
}
\newglossaryentry{gVuex}{
	name=Vuex,
	description={
		Knihovna pro Vue.js, která se stará o state management.
	}
}
\newglossaryentry{gWebpack}{
	name=Webpack,
	description={
		Webpack je software, který zpracovává součásti webových aplikací a tvoří z nich balíčky vhodné pro webové prohlížeče. Primárně je zaměřen na Javascript, ale dokáže zpracovávat i řadu dalších formátů, přes styly v css či sass, obrázky v png, jpeg, svg či konfigurace v json, yaml a dalších.
	}
}
\newglossaryentry{gWebView}{
	name=WebView,
	description={
		Komponenta nativní Android aplikace, která zobrazuje stanovenou URL jako svůj obsah. Používá se zejména v místech, kde je žádoucí zobrazovat obsah z webu, ale je potřeba přístup k funkcím zařízení, ke kterým není možné přístupovat z běžného webového prohlížeče.
	}
}
