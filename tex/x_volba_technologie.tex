\chapter{Volba technologie}\label{technology}

Jelikož bude aplikace rozdělena na backend, kterým se zabývá můj kolega Bc.~Pavel Kovář, a frontend, který je předmětem této práce, je vhodné věnovat jistou část textu volbě vhodné technologie.

\section{Cílová platforma}

Aplikace je navrhována s ohledem na hardwarové vybavení skladu, ve kterém bude poprvé nasazována: zdejší skladníci jsou vybaveni mobilními telefony \emph{Zebra TC20}, které disponují OS Android 7.1 a vestavěnou čtečkou čárových kódů. Kromě skladníků by měla být aplikace použitelná také z tabletu či stolního počítače pro účely vedoucího pracovníka. // TODO přepsat. Z důvodu jednoduchosti vývoje,  testování a aktualizaci bylo hned při úvodním návrhu určeno, že aplikace bude ve formě webové služby, která bude zobrazována ve WebView v jednoduchém kontejneru chovajícím se jako nativní aplikace. // TODO přepsat
Z toho důvodu jsou v následující rešerši zhodnocovány frameworky či knihovny, které usnadňují vývoj \emph{webových aplikací}.

%%%%%%%%%%%%%%%%%%%%%%%%%%%%%%%%%%%%%%%%
%%%%%%%%%%%%%%%%%%%%%%%%%%%%%%%%%%%%%%%%
%%%%%%%%%%%%%%%%%%%%%%%%%%%%%%%%%%%%%%%%
%%%%%%%%%%%%%%%%%%%%%%%%%%%%%%%%%%%%%%%%

\section{Frameworky a knihovny}

V době psaní této práce patří mezi nejpopulárnější \cite{frameworks-github} \cite{frameworks-hackr} front-endové frameworky či knihovny Angular \cite{angular}, React \cite{react}, Vue.js \cite{vue}, Ember.js \cite{ember} a Backbone.js \cite{backbone}.

\paragraph{Názvosloví} Pro účely tohoto textu budu na následujících řádcích používat slovo \emph{framework} při referenci jak frameworků, tak knihoven, z důvodu snížení opakování textu.\\

%%%%%%%%%%%%%%%%%%%%%%%%%%%%%%%%%%%%%%%%
%%%%%%%%%%%%%%%%%%%%%%%%%%%%%%%%%%%%%%%%

\subsection{Datum vydání}

Zatímco v současnosti nejčastěji porovnávanými frameworky jsou první dva zmíněné, Vue.js je z této pětice vybraných nejmladší, nabírá ale velké obliby. Ember.js a Backbone.js jsou poté lehce upozaděny z důvodu jejich stáří. Přehled prvního vydání jednotlivých frameworků je v tabulce \ref{table:compare:release}

\begin{table}[h]
\caption{Volba frameworku: Datum vydání}
\label{table:compare:release}
\begin{tabular}{lrrrrr}
\hline
                                         & Angular                     & React                     & Vue.js                     & Ember.js                     & Backbone.js               \\ \hline
Vydání první verze                       & 2010/2016\footnote{V roce 2010 byl vydán AngularJS, který byl v roce 2016 kompletně přepsán do TypeScriptu a vydán jako Angular 2, či jednoduše \emph{Angular}.}                                                                       & 2013                      & 2014                       & 2011                         & 2010                      \\
\end{tabular}
\end{table}

Datum vydání ovšem nelze objektivně ohodnotit bodovým ziskem. Na jedné straně stojí fakt, že starší framework může být vyspělejší a tudíž stabilnější atp., na straně druhé nové frameworky se často učí z chyb provedených jejich předchůdci a vyberou z nich pouze to nejlepší. Tato tabulka tedy zůstane čistě přehledová.

%%%%%%%%%%%%%%%%%%%%%%%%%%%%%%%%%%%%%%%%
%%%%%%%%%%%%%%%%%%%%%%%%%%%%%%%%%%%%%%%%

\subsection{Počtu hvězdiček na GitHubu}

Počet hvězdiček na GitHubu lze velmi volně interpretovat jako oblíbenost frameworku mezi vývojáři. Z tohoto důvodu již v tabulce \ref{table:compare:github_stars} hodnotím frameworky dle počtu získaných hvězdiček. Hodnocení přeskakuje bodový zisk 3, aby bylo zhodnoceno i absolutní množství hvězdiček, nejen pořadí.

\begin{table}[h]
\caption{Volba frameworku: Počet hvězdiček na GitHubu}
\label{table:compare:github_stars}
\begin{tabular}{lrrrrr}
\hline
                                         & Angular                     & React                     & Vue.js                     & Ember.js                     & Backbone.js               \\ \hline
Počet hvězdiček\\na GitHubu\footnote{Stav k 17. 12. 2018} &   43,6k    & 117,7k                    & 122,3k                     & 20,3k                        & 27,3k                     \\
\makecell[r]{\textit{bodový zisk}}       & \textit{2}                  & \textit{4}                & \textit{5}                 & \textit{0}                   & \textit{1}                  
\end{tabular}
\end{table}

%%%%%%%%%%%%%%%%%%%%%%%%%%%%%%%%%%%%%%%%
%%%%%%%%%%%%%%%%%%%%%%%%%%%%%%%%%%%%%%%%


\subsection{Zázemí}

Zatímco Angular a React jsou vyvíjeny velkými společnostmi: Googlem, respektive Facebookem, které zná každý, Ember.js je vyvíjen společností Tilde Inc. \cite{tilde}, která také není žádným startupem. Vue.js a Backbone.js by se naopak daly nazvat \emph{komunitními projekty}, neboť jsou vytvořeny převážně jedním autorem (Evan You, respektive Jeremy Ashkenas) a rozvíjeny a udržovány komunitou vývojářů. 
\\
Na první pohled by se mohlo zdát, že z tohoto hodnocení budou vycházet lépe ty frameworky, které mají za sebou stabilní firmy, neboť je tím zajištěn jejich kontinuální vývoj. Ve skutečnosti ale velké firmy \emph{zabíjejí} své projekty poměrně často, stačí se podívat například na seznam projektů, které ukončil Google \cite{killed_by_google}. Oproti tomu komunitní projekty mohou žít dále i v případě, že jejich hlavní autor už na projektu nechce, nebo nemůže pracovat. Z toho důvodu nelze jednoznačně určit, které zázemí je pro budoucnost frameworku výhodnější, a u tabulky \ref{table:compare:background} se tedy opět zdržuji udělování bodů.

\begin{table}[h]
\caption{Volba frameworku: Zázemí}
\label{table:compare:background}
\begin{tabular}{lrrrrr}
\hline
                                         & Angular                     & React                     & Vue.js                     & Ember.js                     & Backbone.js               \\ \hline
Zázemí velké\\společnosti                & ano                         & ano                       & ne                         & částečně                     & ne                        \\
\end{tabular}
\end{table}

%%%%%%%%%%%%%%%%%%%%%%%%%%%%%%%%%%%%%%%%
%%%%%%%%%%%%%%%%%%%%%%%%%%%%%%%%%%%%%%%%

\subsection{Křivka učení}

Složitost frameworku je důležitá metrika, neboť má dopady zejména na ekonomickou stránku projektu. Jednoduché prvotní vniknutí do problematiky frameworku ovšem také nemusí být nutně výhodou, pokud v něm je později problémové provést některé pokročilé věci, nebo i v pokročilém stádiu zdržuje svým nízkoúrovňovým přístupem k problémům, které jiné frameworky řeší automaticky.

\paragraph{Angular, React a Vue.js} Přehled obtížnosti tří v současnosti nejčastěji skloňovaných frameworků přehledně shrnul Rajdeep Chandra ve své prezentaci \emph{My experience with Angular 2 , React and Vue} \cite{frameworks-3compare}, ze které vychází hodnocení v tabulce \ref{table:compare:difficulty}.

\paragraph{Ember.js} Tento framework je dle V. Lascika \cite{ember-diffuculty} vhodný spíše pro projekty, na kterých pracuje velké množství vývojářů, a z toho důvodu jej v tabulce \ref{table:compare:difficulty} hodnotím nula body.

\paragraph{Backbone.js} U této knihovny je důležité zmínit, že umožňuje vývojáři vytvořit si strukturu aplikace kompletně dle svého uvážení \cite{frameworks-rubygarage}. To ssebou může nést jak výhody pro zkušeného, tak nevýhody pro nezkušeného vývojáře, který v pokročilém stádiu vývoje může zjistit, že některou ze základních struktur navrhl špatně. Samotná obtížnost práce s touto knihovnou je ale poměrně nízká.

\begin{table}[h]
\caption{Volba frameworku: Obtížnost}
\label{table:compare:difficulty}
\begin{tabular}{lrrrrr}
\hline
                                         & Angular                     & React                     & Vue.js                     & Ember.js                     & Backbone.js               \\ \hline
Obtížnost                                & vysoká                      & vyšší                     & nízká                      & velmi vysoká\footnote{za předpokladu, že na projektu bude pracovat pouze velmi malé množství vývojářů}                                                                                                                                  & nízká                     \\
\makecell[r]{\textit{bodový zisk}}       & \textit{1}                  & \textit{2}                & \textit{4}                 & \textit{1}                   & \textit{4}                  
\end{tabular}
\end{table}

%%%%%%%%%%%%%%%%%%%%%%%%%%%%%%%%%%%%%%%%
%%%%%%%%%%%%%%%%%%%%%%%%%%%%%%%%%%%%%%%%

\subsection{Firemní stack}

Další zvolenou metrikou je, jak daná technologie zapadá do firemní stacku firmy Jagu s.r.o., ve které bude tento nový projekt realizován. Firma se specializuje především na webové aplikace a middlewary na zakázku \cite{jaguweb}, a mezi nejpoužívanější technologie patří PHP (Nette, Laravel, Symfony), dále provozuje jeden informační systém postavený na Angularu a nově také menší aplikaci ve Vue.js. Tabulka \ref{table:compare:stack} shrnuje, jak jsou jednotlivé frameworky blízko k tomuto stacku.

\begin{table}[h]
\caption{Volba frameworku: Shoda s firemním stackem}
\label{table:compare:stack}
\begin{tabular}{lrrrrr}
\hline
                                          & Angular                     & React                     & Vue.js                     & Ember.js                     & Backbone.js               \\ \hline
Shoda s\\firemním stackem                 & ano                         & ne                        & ano                        & ne                           & ne                        \\
\makecell[r]{\textit{bodový zisk}}        & \textit{2}                  & \textit{0}                & \textit{2}                 & \textit{0}                   & \textit{0}                  
\end{tabular}
\end{table}

Angular je zde ohodnocen třemi body, neboť se jedná o framework, který již firma používá. Vue.js je také ohodnoceno jedním bodem, protože vzniklo na základě Angularu a dá se také jednoduše do stávajícího projektu v Angularu integrovat.

%%%%%%%%%%%%%%%%%%%%%%%%%%%%%%%%%%%%%%%%
%%%%%%%%%%%%%%%%%%%%%%%%%%%%%%%%%%%%%%%%

\subsection{Dostupnost vývojářů}

Metrikou, kterou z hlediska udržitelnosti projektu a jeho ekonomických nákladů nelze opomenout, je dostupnost a cena vývojářů se zájmem o danou technologii.
\\
Tato data se ale obtížněji získávají, většina statistik naopak hovoří o nabídkách práce v dané technologii, nikoliv o počtu lidí, kteří s ní pracují. Z toho důvodu jsem se rozhodl založit tuto metriku na výsledcích vyhledávání osob v profesní síti LinkedIn - tak dokážeme zjistit alespoň hrubý počet lidí, kteří o sobě sami tvrdí, že jsou vývojáři v daném frameworku.
\\
Bodové zisky zde hrubě reflektují relativní počet nalezených profilů.

\begin{table}[h]
\caption{Volba frameworku: Počet vývojářů na LinkedIn}
\label{table:compare:developers}
\begin{tabular}{lrrrrr}
\hline
                                          & Angular                     & React                     & Vue.js                     & Ember.js                     & Backbone.js               \\ \hline
Počet výsledků\\na dotaz\\"<název> developer"              & 344k       & 333k                      & 78k                        & 21k                          & 76k                       \\
\makecell[r]{\textit{bodový zisk}}        & \textit{4}                  & \textit{4}                & \textit{2}                 & \textit{1}                   & \textit{2}                  
\end{tabular}
\end{table}

%%%%%%%%%%%%%%%%%%%%%%%%%%%%%%%%%%%%%%%%
%%%%%%%%%%%%%%%%%%%%%%%%%%%%%%%%%%%%%%%%

\subsection{Počet npm balíků}

Npm \cite{npm} je repozitář javascriptových komponent, na kterém jsou sdíleny jednak kompletní řešení (jako například Angular, Rect, Vue.js a další), ale především různé rozšiřující pluginy do těchto frameworků. Z toho důvodu budu v následující metrice hodnotit, kolik balíků npm nabízí pro jednotlivé porovnávané frameworky.
\\
Bodové zisky hrubě odpovídají relativnímu počtu nalezených balíků.

\begin{table}[h]
\caption{Volba frameworku: Počet npm balíků}
\label{table:compare:npm}
\begin{tabular}{lrrrrr}
\hline
                                         & Angular                     & React                     & Vue.js                     & Ember.js                     & Backbone.js               \\ \hline
Počet npm balíků                         & 26,6k                       & 73,4k                     & 20,6k                      & 6,4k                         & 1,5k                      \\
\makecell[r]{\textit{bodový zisk}}       & \textit{2}                  & \textit{3}                & \textit{2}                 & \textit{1}                   & \textit{0}                 
\end{tabular}
\end{table}

%%%%%%%%%%%%%%%%%%%%%%%%%%%%%%%%%%%%%%%%
%%%%%%%%%%%%%%%%%%%%%%%%%%%%%%%%%%%%%%%%

\subsection{Licence}

Licence k použití frameworku je důležitá položka při rozhodování. Naštěstí všech 5 porovnávaných frameworků je v době psaní této licencováno pod MIT licencí, která povoluje jakékoliv použití i v komerční sféře, úpravy, distribuce i použití v ne-opensource projektech. Nevýhodou této licence je nulová záruka funkčnosti či zodpovědnost autorů za potenciální spáchané škody tímto softwarem.
\\
\paragraph{Licencování Reactu} Facebook původně vydal svůj React pod BSD licencí spolu s dalšími patenty, avšak 24. září 2017 byl React převeden pod MIT licenci \cite{react-license-commit, react-license}.
\\
Jelikož jsou všechny frameworky licencovány stejně, neprobíhá v tabulce \ref{table:compare:license} žádné bodování.

\begin{table}[h]
\caption{Volba frameworku: Licence}
\label{table:compare:license}
\begin{tabular}{lrrrrr}
\hline
                                         & Angular                     & React                     & Vue.js                     & Ember.js                     & Backbone.js               \\ \hline
Licence                                  & MIT                         & MIT                       & MIT                        & MIT                          & MIT                       \\
\end{tabular}
\end{table}

%%%%%%%%%%%%%%%%%%%%%%%%%%%%%%%%%%%%%%%%
%%%%%%%%%%%%%%%%%%%%%%%%%%%%%%%%%%%%%%%%

\subsection{Vývojářské nástroje}

Dalším důležitým nástrojem při práci s frameworkem je možnost jeho debuggování. Framework by měl nabízet vlastní řešení, které vývojáři usnadní nalézt chybu, zjistit, jak se jeho kód chová či odladil rychlostní problémy.
\\
Všechny ze zde porovnávaných frameworků nabízejí tyto nástroje formou doplňku do prohlížeče, konkrétně se dále budeme bavit o doplňcích do Google Chrome.

\begin{table}[h]
\caption{Volba frameworku: Devtools}
\label{table:compare:devtools}
\begin{tabular}{lrrrrr}
\hline
                                         & Angular                     & React                     & Vue.js                     & Ember.js                     & Backbone.js               \\ \hline
Název                                    & Augury    & \makecell[r]{React\\Developer\\Tools} & \makecell[r]{Vue.js\\devtools} & \makecell[r]{Ember\\inspector} & \makecell[r]{Backbone\\Debugger} \\
Počet stažení\footnote{Stav k 24. 12. 2018}  & 230k                    & 1.351k                    & 706k                       & 57k                          & 9,5k                      \\
\makecell[r]{\textit{bodový zisk}}       & \textit{2}                  & \textit{4}                & \textit{3}                 & \textit{1}                   & \textit{0}                \\
Hodnocení (z 5)                          & 3.9                         & 4.2                       & 4.7                        & 4.8                          & 4,5                       \\
\makecell[r]{\textit{bodový zisk}}       & \textit{2}                  & \textit{1}                & \textit{3}                 & \textit{4}                   & \textit{2}                  
\end{tabular}
\end{table}

%%%%%%%%%%%%%%%%%%%%%%%%%%%%%%%%%%%%%%%%
%%%%%%%%%%%%%%%%%%%%%%%%%%%%%%%%%%%%%%%%

\subsection{Oficiální dokumentace}

Hlavním zdrojem ke studiu frameworku by měla být jeho oficiální dokumentace, v této sekci tedy budu hodnotit kvalitu a obsáhlost oficiálního manuálu k jednotlivým frameworkům.

\paragraph{Angular} Jedná se o velmi obsáhlou a dobře rozdělenou dokumentaci \cite{angular-doc}, která obsahuje i řadu příkladů a ve srozumitelné stromové struktuře vývojář jednoduše najde, co potřebuješ.

\paragraph{React} Dokumentace Reactu \cite{react-doc} je o poznání jednodušší než ta Angularu, avšak to je způsobeno tím, že React je pouze knihovna, kdežto Angular je plnohodnotný framework. Dokumentace je rozdělena na jednodušší úvod a pokročilejší techniky, je tedy snadné s ní pracovat.

\paragraph{Vue.js} Nejmladší z frameworků má také velmi přátelskou dokumentaci \cite{vue-doc}, která je podobně jako u Angularu velmi bohatá a stromově strukturovaná.

\paragraph{Ember.js} Oficiální manuál Ember.js \cite{ember-doc} je taktéž poměrně obsáhlý a strukturou připomíná dokumentaci Angularu a Vue.js. Obsahuje velké množství ukázek kódu a je logicky strukturován.

\paragraph{Backbone.js} Poslední ze zkoumaných frameworků má oficiální dokumentaci \cite{backbone-doc} na první pohled méně atraktivní a pro nováčka může být matoucí. Oproti ostatním dokumentacím chybí například barevné zvýraznění důležitých bodů a další grafické strukturování textu.

\begin{table}[h]
\caption{Volba frameworku: Dokumentace}
\label{table:compare:docs}
\begin{tabular}{lrrrrr}
\hline
                                         & Angular                     & React                     & Vue.js                     & Ember.js                     & Backbone.js               \\ \hline
\makecell{Kvalita oficiální\\dokumentace} & \makecell{velmi\\vysoká}   & vysoká                    & \makecell{velmi\\vysoká}   & \makecell{velmi\\vysoká}     & střední                   \\
\makecell[r]{\textit{bodový zisk}}       & \textit{3}                  & \textit{2}                & \textit{3}                 & \textit{3}                   & \textit{1}                  
\end{tabular}
\end{table}

%%%%%%%%%%%%%%%%%%%%%%%%%%%%%%%%%%%%%%%%
%%%%%%%%%%%%%%%%%%%%%%%%%%%%%%%%%%%%%%%%

\subsection{Otázky na Stack Overflow}

Stack Overflow je jedním z portálů sítě Stack Exchange, který zná prakticky každý vývojář. Kdokoliv zde může položit otázku a komunita poté odpovídá, zatímco hlasuje o kvalitě odpovědí, aby byla vybrána ta nejlepší.\\
Z pohledu volby frameworku může být na jednu stranu vhodné, aby bylo na této stránce hodně otázek týkajících se dané technologie, na druhou stranu to ale může znamenat i nekvalitní dokumentaci. Jelikož ale v předchozí sekci nebyla žádná dokumentace vyhodnocena jako vysloveně špatná, budu dále usuzovat, že větší množství otázek je lepší.

\begin{table}[h]
\caption{Volba frameworku: Otázky na Stack Overflow}
\label{table:compare:stackoverflow}
\begin{tabular}{lrrrrr}
\hline
                                         & Angular                     & React                     & Vue.js                     & Ember.js                     & Backbone.js               \\ \hline
\makecell[l]{Počet otázek\\na Stack Overflow} & 146k                   & 118k                      & 28k                        & 23k                          & 21k                       \\
\makecell[l]{Počet \emph{zodpovězených}\\otázek} & 86k                 & 72k                       & 18k                        & 17k                          & 16k                       \\
\makecell[r]{\textit{bodový zisk}}       & \textit{3}                  & \textit{3}                & \textit{1}                 & \textit{1}                   & \textit{1}                  
\end{tabular}
\end{table}

%%%%%%%%%%%%%%%%%%%%%%%%%%%%%%%%%%%%%%%%
%%%%%%%%%%%%%%%%%%%%%%%%%%%%%%%%%%%%%%%%

\subsection{Integrace se Sentry}

Sentry \cite{sentry} je nástroj sloužící k automatickému i manuálnímu záznamu chyb v aplikacích. Ve firmě Jagu s.r.o. je využíván v řadě projektů a jeho nasazení bude vhodné i pro aplikaci řešenou v rámci této práce. Z toho důvodu je vhodné se podívat, jak hlubokou integraci je možné mezi jednotlivými frameworky a Sentry realizovat.\\
Při pohledu na přehled toho, jaké technologie Sentry podporuje v JavaScriptu \cite{sentry-js} rychle zjišťujeme, že všech pět zde zkoumaných frameworků je oficiálně podporováno, včetně rychlého návodu na zprovoznění. Z toho důvodu neprobíhá v tabulce \ref{table:compare:sentry} žádné bodování.

\begin{table}[h]
\caption{Volba frameworku: Integrace se Sentry}
\label{table:compare:sentry}
\begin{tabular}{lrrrrr}
\hline
                                         & Angular                     & React                     & Vue.js                     & Ember.js                     & Backbone.js               \\ \hline
\makecell[l]{Oficiální integrace\\se Sentry} & ano                      & ano                       & ano                        & ano                          & ano                       \\
\end{tabular}
\end{table}


%%%%%%%%%%%%%%%%%%%%%%%%%%%%%%%%%%%%%%%%
%%%%%%%%%%%%%%%%%%%%%%%%%%%%%%%%%%%%%%%%

% \subsection{Vybavenost frameworku TODO}

% V poslední části se zaměříme na to, jaké funkce jednotlivé frameworky nabízí, a zda je vhodné jejich použití pro frontend skladového systému, který je předmětem této práce. Jelikož systém bude vždy napojen na backend, který bude poskytovat data i bussiness logiku, je vhodné se zamyslet i nad tím, aby framework zbytečně neobsahoval nepotřebné součásti, které by v tomto případě byly spíše na obtíž a zanášely jednoduchou aplikaci zbytečnými složitostmi.

% \paragraph{Angular} Ze všech zde porovnávaných frameworků se jedná o ten nejkomplexnější. Nabízí Dependency Injection, Services, Router, TODO

% \paragraph{React} Tím, že je React pouze knihovna a nikoliv framework, nenabízí tolik součástí jako Angular: neexistuje zde Dependency Injection\footnote{DI je do Reactu možné doplnit pomocí knihoven třetích stran}, Services, Router\footnote{Router je možné doplnit knihovnou}, vše je komponenta.TODO

% \paragraph{Vue.js} Co se nabízených komponent týká, Vue.js stojí mezi Angularem a Reactem. Vue.js je považováno za framework, a samo o sobě říká, že obsahuje \emph{to nejlepší z Angularu}. Například Router je možné použít jak oficiální vestavěný, tak jiný router třetí strany \cite{vue-routers}. Dependency injection je také dostupné jako knihovna třetí strany.TODO

% \paragraph{Ember.js} 

% \paragraph{Backbone.js} Backbone.js je z pěti porovnávaných 

%%%%%%%%%%%%%%%%%%%%%%%%%%%%%%%%%%%%%%%%
%%%%%%%%%%%%%%%%%%%%%%%%%%%%%%%%%%%%%%%%

\subsection{Souhrn průzkumu}

V tabulce \ref{table:compare:results} jsou sečteny body z předchozích dílčích hodnocení.

\begin{table}[h]
\caption{Volba frameworku: Výsledky}
\label{table:compare:results}
\begin{tabular}{lrrrrr}
\hline
                                          & Angular                     & React                     & Vue.js                     & Ember.js                     & Backbone.js               \\ \hline
\makecell[r]{bodový zisk celkem}          & 21                          & 23                        & 25                         & 12                           & 11                  
\end{tabular}
\end{table}

Výsledky rozdělují frameworky na dvě skupiny. V té vedoucí je trojice Angular, React a Vue.js, v pozadí poté zůstávají Ember.js a Backbone.\\
První tři frameworky jsou seřazeny poměrně těsně za sebou, avšak nejlépe vyšel ze srovnání nejmladší Vue.js, který tímto volím jako framework, ve kterém budu na práci dále pracovat.
