\chapter{Zpracovaný výstup z uživatelského testování alfa verze aplikace} \label{ap:testing_requirements}

\paragraph{Seznam požadavků}

\begin{itemize}
	\item Expedice:
	\begin{itemize}
		\item přidat možnost zadat váhu balíku,
		\item upozorňovat, že dělám balení skladové položky, která je i v jiné expedici, která byla ale zadána dřív a je ještě nezpracovaná (FIFO),
	\end{itemize}
	\item možnost dokončit vyskladnění částečně, tj. nedokončené položky by se vložily do nového úkolu,
	\item do webview aplikace přidat možnost rozsvítit světlo blesku přímo z aplikace,
	\item mít možnost u skladové položky vyplnit jednak oficiální název ale také skladový název, podle kterého se např. bude lépe položka hledat na skladě,
	\item přehled produktů může být moc dlouhý, když jich je na skladě hodně - rozumně omezit,
	\item k \uv{může mít šarži} přidat vysvětlení (nápovědu),
	\item mít možnost přidávat nové položky i pod tabulkou, které je vypisují,
	\item přidávání čárových kódů skladových položek by mělo být i v šipce vpravo dole, výchozí kódy rovnou v tabulce, ne až na prokliku. Stejná tabulka by mohla být i v úpravě produktu,
	\item při naskladňování položky bez fotky nabídnout, zda chce skladník fotografii přidat,
	\item při kolizi čárového kódu umožnit zobrazení kolizní položky,
	\item umožnit vlastní řazení sloupců na přehledu na domovské obrazovce,
	\item ikonku čárového kódu otočit o 90°,
	\item vedle šipky vpravo dole přidat nadpis \uv{nový úkol},
	\item ze seznamu umístění ve skladu mít akci, která zobrazí položky na konkrétním umístění,
	\item možnost nastavit si pořadí tvorby úkolů v šipce vpravo dole,
	\item umožnit zobrazit nahranou přílohu,
	\item umožnit nahrát více souborů postupně,
	\item umožnit měnit prioritu úkolu i po jeho vytvoření,
	\item v naskladnění zpřístupnit i přehled podle produktů a kam jsou naskladněny, ne pouze přehled umístění, a co na ně bylo umístěno,
	\item při dokončování úkolů mít možnost přiřadit i fotku,
	\item při volbě položek k vyskladnění či přesunu vypisovat jejich stručné informace ve volném místě vpravo,
	\item u umístění mít možnost evidovat, zda se na něm dá provádět expedice,
	\item snack zpráva o nově vytvořeném úkolu by měla obsahovat jeho ID a odkaz na něj,
	\item seznam umístění, kde lze požadovanou položku nalézt, řadit pod sebe, pouze jedno na řádek,
	\item přidat na bottom sheet křížek pro jeho zavření,
	\item automatické obnovování seznamů na domovské obrazovce,
	\item možnost zvolit komu úkol přiřadím, skupiny skladníků,
	\item po navrácení na domovskou stránku ponechat aktivní tu záložku, ze které byl úkol otevřený,
	\item při manipulaci se skladovými položkami v úkolech umožnit změnu jejich počtu i rozklikem položky, ne nutně napípnutím,
	\item při naskladňování přidat volitelně možnost použití skladníkova inventáře, aby viděl, co ještě musí umístit,
	\item při výběru umístění ve formulářích vypisovat i jeho kód
	\item při výběru položek k vyskladnění či přesunu umožnit je přidávat kliknutím na jejich název (ve výpisu \uv{položek na skladu},
	\item potom, co skladník zvolí, že bude pracovat na úloze, přepnout rovnou na druhou záložku úkolu,
	\item více zvýraznit když se seznam úkolů načítá,
	\item informace o tom, co by se mělo skenovat, by měla být nejen pod vstupním polem, ale i přímo v něm,
	\item možnost aktivovat umístění, na které naskladňuji, pouhým kliknutím na něj, bez nutnosti skenovat kód,
	\item při přesunu položek na cílové umístění povolit rychlou akci, která přesune vše.
\end{itemize}

\paragraph{Seznam chyb}

\begin{itemize}
	\item Název umístění v přehledu hotových úloh je moc nevýrazný,
	\item překlopení Zebry na bok ukončí aktivitu - aplikace se poté musí znovu přihlásit,
	\item ve webview nefunguje otevření jiné přílohy než png, jpg,
	\item hláška o načítání na umístění je nejasná: mělo by být spíš \uv{Načtěte produkty k umístění do ...},
	\item když mám přesouvat produkt, který už je i cílovém umístění, tak stále vypisuje chybu a nelze přesunout.
\end{itemize}

\paragraph{Návrhy na pokročilé funkce}

\begin{itemize}
	\item Umožnit zadat termíny, kdy pravidelně přijíždí konkrétní dopravce přebírající zásilky. Díky tomu by mohl systém upřednostňovat expedici podle termínu, kdy musí být balíky připravené.
	\item Umožnit současně pracovat na více úkolech.
	\item Preferování různých umístění při naskladnění: do zcela volných umístění apod.
	\item Podpora pro výměnu vratných obalů (sklenice, sodastream, apod.).
	\item Vedoucímu umožnit nastavit k výrobku speciální upozornění pro skladníka, které se mu zobrazí (např. zkontroluj, že je na krabici nějaký nápis apod.).
	\item Notifikace na nové úkoly.
	\item Vratky zboží, řešit především návaznosti šarží - tj. trvanlivost apod.,
	\item hromadná avíza: umožnit zadání, které zboží přijede a bude naskladněno. To ale bude najíždět postupně, a mělo by být vidět, co vlastně ještě chybí a má přijet.
	\item Umožnit skladníkovi zadávat si mikro-úkoly například na expedici, když se mu něco na umístění nezdá a chce to zkontrolovat.
	\item Kapacity umístění.
\end{itemize}
